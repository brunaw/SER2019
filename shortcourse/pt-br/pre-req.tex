\documentclass[]{article}
\usepackage{lmodern}
\usepackage{amssymb,amsmath}
\usepackage{ifxetex,ifluatex}
\usepackage{fixltx2e} % provides \textsubscript
\ifnum 0\ifxetex 1\fi\ifluatex 1\fi=0 % if pdftex
  \usepackage[T1]{fontenc}
  \usepackage[utf8]{inputenc}
\else % if luatex or xelatex
  \ifxetex
    \usepackage{mathspec}
  \else
    \usepackage{fontspec}
  \fi
  \defaultfontfeatures{Ligatures=TeX,Scale=MatchLowercase}
\fi
% use upquote if available, for straight quotes in verbatim environments
\IfFileExists{upquote.sty}{\usepackage{upquote}}{}
% use microtype if available
\IfFileExists{microtype.sty}{%
\usepackage{microtype}
\UseMicrotypeSet[protrusion]{basicmath} % disable protrusion for tt fonts
}{}
\usepackage[margin=1in]{geometry}
\usepackage{hyperref}
\hypersetup{unicode=true,
            pdftitle={Análise de dados musicais no R},
            pdfauthor={Bruna Wundervald \& Julio Trecenti},
            pdfborder={0 0 0},
            breaklinks=true}
\urlstyle{same}  % don't use monospace font for urls
\usepackage{color}
\usepackage{fancyvrb}
\newcommand{\VerbBar}{|}
\newcommand{\VERB}{\Verb[commandchars=\\\{\}]}
\DefineVerbatimEnvironment{Highlighting}{Verbatim}{commandchars=\\\{\}}
% Add ',fontsize=\small' for more characters per line
\usepackage{framed}
\definecolor{shadecolor}{RGB}{248,248,248}
\newenvironment{Shaded}{\begin{snugshade}}{\end{snugshade}}
\newcommand{\AlertTok}[1]{\textcolor[rgb]{0.94,0.16,0.16}{#1}}
\newcommand{\AnnotationTok}[1]{\textcolor[rgb]{0.56,0.35,0.01}{\textbf{\textit{#1}}}}
\newcommand{\AttributeTok}[1]{\textcolor[rgb]{0.77,0.63,0.00}{#1}}
\newcommand{\BaseNTok}[1]{\textcolor[rgb]{0.00,0.00,0.81}{#1}}
\newcommand{\BuiltInTok}[1]{#1}
\newcommand{\CharTok}[1]{\textcolor[rgb]{0.31,0.60,0.02}{#1}}
\newcommand{\CommentTok}[1]{\textcolor[rgb]{0.56,0.35,0.01}{\textit{#1}}}
\newcommand{\CommentVarTok}[1]{\textcolor[rgb]{0.56,0.35,0.01}{\textbf{\textit{#1}}}}
\newcommand{\ConstantTok}[1]{\textcolor[rgb]{0.00,0.00,0.00}{#1}}
\newcommand{\ControlFlowTok}[1]{\textcolor[rgb]{0.13,0.29,0.53}{\textbf{#1}}}
\newcommand{\DataTypeTok}[1]{\textcolor[rgb]{0.13,0.29,0.53}{#1}}
\newcommand{\DecValTok}[1]{\textcolor[rgb]{0.00,0.00,0.81}{#1}}
\newcommand{\DocumentationTok}[1]{\textcolor[rgb]{0.56,0.35,0.01}{\textbf{\textit{#1}}}}
\newcommand{\ErrorTok}[1]{\textcolor[rgb]{0.64,0.00,0.00}{\textbf{#1}}}
\newcommand{\ExtensionTok}[1]{#1}
\newcommand{\FloatTok}[1]{\textcolor[rgb]{0.00,0.00,0.81}{#1}}
\newcommand{\FunctionTok}[1]{\textcolor[rgb]{0.00,0.00,0.00}{#1}}
\newcommand{\ImportTok}[1]{#1}
\newcommand{\InformationTok}[1]{\textcolor[rgb]{0.56,0.35,0.01}{\textbf{\textit{#1}}}}
\newcommand{\KeywordTok}[1]{\textcolor[rgb]{0.13,0.29,0.53}{\textbf{#1}}}
\newcommand{\NormalTok}[1]{#1}
\newcommand{\OperatorTok}[1]{\textcolor[rgb]{0.81,0.36,0.00}{\textbf{#1}}}
\newcommand{\OtherTok}[1]{\textcolor[rgb]{0.56,0.35,0.01}{#1}}
\newcommand{\PreprocessorTok}[1]{\textcolor[rgb]{0.56,0.35,0.01}{\textit{#1}}}
\newcommand{\RegionMarkerTok}[1]{#1}
\newcommand{\SpecialCharTok}[1]{\textcolor[rgb]{0.00,0.00,0.00}{#1}}
\newcommand{\SpecialStringTok}[1]{\textcolor[rgb]{0.31,0.60,0.02}{#1}}
\newcommand{\StringTok}[1]{\textcolor[rgb]{0.31,0.60,0.02}{#1}}
\newcommand{\VariableTok}[1]{\textcolor[rgb]{0.00,0.00,0.00}{#1}}
\newcommand{\VerbatimStringTok}[1]{\textcolor[rgb]{0.31,0.60,0.02}{#1}}
\newcommand{\WarningTok}[1]{\textcolor[rgb]{0.56,0.35,0.01}{\textbf{\textit{#1}}}}
\usepackage{graphicx,grffile}
\makeatletter
\def\maxwidth{\ifdim\Gin@nat@width>\linewidth\linewidth\else\Gin@nat@width\fi}
\def\maxheight{\ifdim\Gin@nat@height>\textheight\textheight\else\Gin@nat@height\fi}
\makeatother
% Scale images if necessary, so that they will not overflow the page
% margins by default, and it is still possible to overwrite the defaults
% using explicit options in \includegraphics[width, height, ...]{}
\setkeys{Gin}{width=\maxwidth,height=\maxheight,keepaspectratio}
\IfFileExists{parskip.sty}{%
\usepackage{parskip}
}{% else
\setlength{\parindent}{0pt}
\setlength{\parskip}{6pt plus 2pt minus 1pt}
}
\setlength{\emergencystretch}{3em}  % prevent overfull lines
\providecommand{\tightlist}{%
  \setlength{\itemsep}{0pt}\setlength{\parskip}{0pt}}
\setcounter{secnumdepth}{0}
% Redefines (sub)paragraphs to behave more like sections
\ifx\paragraph\undefined\else
\let\oldparagraph\paragraph
\renewcommand{\paragraph}[1]{\oldparagraph{#1}\mbox{}}
\fi
\ifx\subparagraph\undefined\else
\let\oldsubparagraph\subparagraph
\renewcommand{\subparagraph}[1]{\oldsubparagraph{#1}\mbox{}}
\fi

%%% Use protect on footnotes to avoid problems with footnotes in titles
\let\rmarkdownfootnote\footnote%
\def\footnote{\protect\rmarkdownfootnote}

%%% Change title format to be more compact
\usepackage{titling}

% Create subtitle command for use in maketitle
\providecommand{\subtitle}[1]{
  \posttitle{
    \begin{center}\large#1\end{center}
    }
}

\setlength{\droptitle}{-2em}

  \title{Análise de dados musicais no \texttt{R}}
    \pretitle{\vspace{\droptitle}\centering\huge}
  \posttitle{\par}
  \subtitle{Pré-Requisitos}
  \author{Bruna Wundervald \& Julio Trecenti}
    \preauthor{\centering\large\emph}
  \postauthor{\par}
    \date{}
    \predate{}\postdate{}
  

\begin{document}
\maketitle

Bem-vindos! Este documento contém as informações sobre os pré-requisitos
do minicurso ``Análise de dados musicais no \texttt{R}'', a ser
ministrado durante o IV SER, em Niterói, RJ.

Espera-se que os participantes já tenham instalados em seu computador o
\texttt{R} e o RStudio (ou outra IDE de preferência):

\begin{itemize}
\tightlist
\item
  \url{https://cran.r-project.org/doc/manuals/r-release/R-admin.html}
\item
  \url{https://www.rstudio.com/}
\end{itemize}

As versões não devem fazer muita diferença, contanto que os pacotes
descritos abaixo possam ser instalados.

\hypertarget{pacotes}{%
\subsection{Pacotes}\label{pacotes}}

Os seguintes pacotes serão usados no decorrer do minicurso:

\begin{itemize}
\tightlist
\item
  \texttt{tidyverse}
\item
  \texttt{chorrrds}
\item
  \texttt{vagalumeR}
\item
  \texttt{Rspotify}
\item
  \texttt{tidytext}
\item
  \texttt{tm}
\item
  \texttt{lexiconPT}
\item
  \texttt{ggridges}
\item
  \texttt{chorddiag}: \url{https://github.com/mattflor/chorddiag}
\end{itemize}

Fora o último, todos estes pacotes estão no CRAN e não devem apresentar
maiores problemas para a instalação. Para instalar, uma opção é:

\begin{Shaded}
\begin{Highlighting}[]
\CommentTok{# Pacotes do CRAN}
\NormalTok{pacotes <-}\StringTok{ }\KeywordTok{c}\NormalTok{(}\StringTok{"tidyverse"}\NormalTok{, }\StringTok{"chorrrds"}\NormalTok{, }\StringTok{"vagalumeR"}\NormalTok{, }\StringTok{"Rspotify"}\NormalTok{, }
             \StringTok{"tidytext"}\NormalTok{, }\StringTok{"tm"}\NormalTok{, }\StringTok{"lexiconPT"}\NormalTok{, }\StringTok{"ggridges"}\NormalTok{,}\StringTok{"devtools"}\NormalTok{)}

\NormalTok{inst <-}\StringTok{  }\KeywordTok{try}\NormalTok{(}\KeywordTok{lapply}\NormalTok{(pacotes, library, }\DataTypeTok{character.only =} \OtherTok{TRUE}\NormalTok{),}
             \DataTypeTok{silent =} \OtherTok{TRUE}\NormalTok{)}
  
\NormalTok{faltante <-}\StringTok{  }\KeywordTok{which}\NormalTok{(}\KeywordTok{is.na}\NormalTok{(}\KeywordTok{match}\NormalTok{(pacotes,}\KeywordTok{installed.packages}\NormalTok{()[,}\StringTok{'Package'}\NormalTok{])))}

\ControlFlowTok{if}\NormalTok{(}\KeywordTok{length}\NormalTok{(faltante)}\OperatorTok{>}\DecValTok{0}\NormalTok{) \{}
  \KeywordTok{cat}\NormalTok{(}\StringTok{'Alguns pacotes estão faltando. Instalando...}\CharTok{\textbackslash{}n}\StringTok{'}\NormalTok{)}
  \KeywordTok{cat}\NormalTok{(pacotes[faltante], }\StringTok{'}\CharTok{\textbackslash{}n}\StringTok{'}\NormalTok{)}
  \KeywordTok{lapply}\NormalTok{(pacotes[faltante], install.packages)}
\NormalTok{\}}

\CommentTok{# chorddiag}
\NormalTok{devtools}\OperatorTok{::}\KeywordTok{install_github}\NormalTok{(}\StringTok{"mattflor/chorddiag"}\NormalTok{)}
\end{Highlighting}
\end{Shaded}

\hypertarget{nocoes}{%
\subsection{Noções}\label{nocoes}}

Este minicurso requer algumas noções básicas sobre \texttt{R},
expecialmente sobre os pacotes do \texttt{tidyverse}. Fora os pacotes
epecíficos, nós utilizaremos intensivamente também:

\begin{itemize}
\tightlist
\item
  \texttt{dplyr}
\item
  \texttt{stringr}
\item
  \texttt{purrr}
\item
  \texttt{ggplot2}
\end{itemize}

Para quem precisar dar uma relembrada no que estes pacotes fazem, nós
recomendamos os materiais:

\begin{itemize}
\tightlist
\item
  \url{https://www.curso-r.com/material/}
\item
  \url{https://www.curso-r.com/blog/}
\end{itemize}

De qualquer forma, as funções utilizadas ao longo do material serão
sempre explicadas.


\end{document}
